% Manual of pgf-umlsd.sty, a convenient set of macros for drawing UML
% sequence diagrams.
% Written by Xu Yuan <xuyuan.cn@gmail.com> from
% Southeast University, China.
% This file is part of pgf-umlsd
% you may get it at http://code.google.com/p/pgf-umlsd/

\documentclass{article}
\usepackage[margin=12mm]{geometry}
\usepackage{hyperref}

\usepackage{tikz}
\usepackage[underline=true,rounded corners=false]{pgf-umlsd}

%%%%%%%%%%%%%%%%%%%%%%%%%%%%%%%%%%%%%%%%%%%%%%%%%%%%%%%%%%%%%%%%%
\usepackage{listings}
\usepackage{color}
\definecolor{listinggray}{gray}{0.92}
\lstset{ %
language=[LaTeX]TeX,
breaklines=true,
frame=single,
% frameround=tttt,
basicstyle=\footnotesize\ttfamily,
backgroundcolor=\color{listinggray},
keywordstyle=\color{blue}
}
%%%%%%%%%%%%%%%%%%%%%%%%%%%%%%%%%%%%%%%%%%%%%%%%%%%%%%%%%%%%%%%%%

%%%%%%%%%%%%%%%%%%%%%%%%%%%%%%%%%%%%%%%%%%%%%%%%%%%%%%%%%%%%%%%%%
\hypersetup{
  colorlinks=true,
  linkcolor=blue,
  anchorcolor=black,
  citecolor=olive,
  filecolor=magenta,
  menucolor=red,
  urlcolor=blue
}
%%%%%%%%%%%%%%%%%%%%%%%%%%%%%%%%%%%%%%%%%%%%%%%%%%%%%%%%%%%%%%%%%

%%%%%%%%%%%%%%%%%%%%%%%%%%%%%%%%%%%%%%%%%%%%%%%%%%%%%%%%%%%%%%%%%
\newcommand{\demo}[2][1]{
\begin{minipage}{.49\linewidth}
\centering
\resizebox{#1\linewidth}{!}{
\input{demo/#2}
}
\end{minipage}
\hspace{0.01\linewidth}
\begin{minipage}{.5\linewidth}
\lstinputlisting{demo/#2}
\end{minipage}
}
%%%%%%%%%%%%%%%%%%%%%%%%%%%%%%%%%%%%%%%%%%%%%%%%%%%%%%%%%%%%%%%%%

%%%%%%%%%%%%%%%%%%%%%%%%%%%%%%%%%%%%%%%%%%%%%%%%%%%%%%%%%%%%%%%%%
\newcommand{\example}[2][1]{
  \begin{center}  
    \resizebox{#1\linewidth}{!}{
      \input{demo/#2}
    }
  \end{center}
  \lstinputlisting{demo/#2}
}
%%%%%%%%%%%%%%%%%%%%%%%%%%%%%%%%%%%%%%%%%%%%%%%%%%%%%%%%%%%%%%%%% 

\begin{document}
%%%%%%%%%%%%%%%%%%%%%%%%%%%%%%%%%%%%%%%%%%%%%%%%%%%%%%%%%%%%%%%%%
\title{Drawing UML Sequence Diagram by using \texttt{pgf-umlsd}}
\author{\href{mailto:xuyuan.cn@gmail.com}{Yuan Xu}}
\date{\today{}~(v0.6)}
\maketitle
%%%%%%%%%%%%%%%%%%%%%%%%%%%%%%%%%%%%%%%%%%%%%%%%%%%%%%%%%%%%%%%%%

\begin{abstract}
  \texttt{pgf-umlsd} is a LaTeX package for drawing UML Sequence
  Diagrams. As stated by its name, it is based on a very popular
  graphic package \texttt{PGF/TikZ}. This document presents the usage
  of \texttt{pgf-umlsd} and collects some UML sequence diagrams as
  examples. \texttt{pgf-umlsd} can be downloaded from
  \href{http://code.google.com/p/pgf-umlsd/}{http://code.google.com/p/pgf-umlsd/}.
\end{abstract}

\tableofcontents

\section{The Essentials}
\subsection{Basic graphics objects}
\subsubsection{empty diagram}
\demo{empty}

\subsubsection{thread}
\demo[0.3]{thread}

\subsubsection{instance}
\demo[0.3]{instance}

\subsubsection{distance between threads and instances}
\demo{distance}

\subsubsection{customization}
The package has two options for customization: \texttt{underline} and
\texttt{rounded corners}, further customization see the example below:

\demo{customize}


\subsection{Call}
\subsubsection{call}
\demo[0.6]{call}

\subsubsection{call self}
\demo[0.6]{callself}

\subsubsection{message call}
\demo[0.6]{messcall}

\subsubsection{nested call}
\demo[0.6]{nested-call}

\subsection{Message}
\demo[0.6]{message}

\subsection{Block}
\demo[0.6]{block}

\section{Examples}
\subsection{Single thread}
\example[0.8]{single-thread-example}

\subsection{Multi-threads}
\example[0.8]{multi-threads-example}

\section{Acknowledgements}
Many people contributed to \texttt{pgf-umlsd} by reporting problems,
suggesting various improvements or submitting code. Here is a list of
these people:
\href{mailto:nobel1984@gmail.com}{Nobel Huang},
\href{mailto:humbert@uni-wuppertal.de}{Dr. Ludger Humbert},
\href{mailto:MathStuf@gmail.com}{MathStuf},
\href{mailto:vlado.handziski@gmail.com}{Vlado Handziski},
and \href{mailto:frankmorgner@gmail.com}{Frank Morgner}.

\end{document}
%%% Local Variables: 
%%% mode: Tex-PDF
%%% TeX-master: t
%%% End: 
